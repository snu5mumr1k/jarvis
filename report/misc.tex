Я справился начать собирать мысли воедино.
Собрал кое-какое вступление (приложу pdf-кой)
После этого хочу вкратце рассказать про:
\begin{list}{}{}
    \item Заинтересованность крупных компаний и их решения
        \begin{list}{}{}
            \item https://www.cisco.com/c/ru\_ru/solutions/internet-of-things/overview.html
            \item https://www.amazon.com/smart-home-devices/b?ie=UTF8\&node=6563140011
        \end{list}
    \item Возможности, устройство и структуру фрейморка от Eclipse
    \item Хранение информации об event'ах и правилах
    (возможно, правильно упомянуть про использование онтологий - www.mdpi.com/1424-8220/17/7/1586/pdf 
    думаю, что, так или иначе, мне придётся реализовать и эту часть)
    \item Что конкретно хочется сделать:
    Прототип реальной системы с небольшим числом датчиков на базе Arduino
    (https://www.arduino.cc/)
    В котором будут реализованы два датчика температуры, регулярно отправляющие информацию о температуре окружающей среды и воды в чайнике (или ванной),
    имитация включателя системы отопления, света и включения чайника (воды в ванной). Поверх этих заглушек будет показан простейший алгоритм работы умного дома.
\end{list}
