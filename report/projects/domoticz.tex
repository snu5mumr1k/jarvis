\subsubsection{Domoticz}

Это кросплатформенный проект, написанный на C++, разрабатываемый сообществом с 2012 года \cite{DomoticzGitHub}.
Система имеет веб-интерфейс для управления умным домом. Устройства могут подключаться как по сети, так и через USB.
Есть возможность написания дополнительных модулей и скриптов. Скрипт на lua будет выполнен с помощью JIT-компиляции (Just In Time),
также можно писать скрипты на других языках, но они будут запущены как внешние процессы.

Достоинствами системы умного дома, построенного с помощью Domoticz являются:
\begin{list}{}{}
    \item Реализован графический интерфейс
    \item Есть достаточно подробные инструкции по установке на многие платформы (Raspberry PI, Windows, LINUX, MacOS, FreeNAS)
    \item Также есть мануалы по подключению устройств и написанию скриптов
    \item У проекта большое сообщество -- есть, где просить помощь в настройке
\end{list}
Недостатками можно назвать:
\begin{list}{}{}
    \item Система уже не молодая и потому достаточно сложно устроена
    \item Требует определённых технических навыков для настройки (например, базового понимания языков программирования)
    \item Поддержка устройств в коде осуществляется явно -- поддержка нового устройства не очень быстрый процесс
\end{list}
