\subsubsection{OpenHAB}

Система разработана как "улучшение существующих решений" \cite{OpenHABDoc}.
Название расшифровывается как \textbf{open} \textbf{H}ome \textbf{A}utomation \textbf{B}us.
Проект больше подходит для людей, далёких от программирования, и имеет подробную инструкцию по установке.
(например, в статье \cite[Keeping Eyes on your Home]{OpenHABArticle}
В инструкции также описаны основные концепты системы, чтобы дать пользователю представление, о том, как всё устроено в коде.

Основана на Eclipse SmartHome, написанном на Java.

Достоинствами системы умного дома, построенного с помощью openHAB являются:
\begin{list}{}{}
    \item Простота настройки
    \item Отсутствие зависимости от платформы
    \item Простой встроенный язык для описания правил управления устройствами
\end{list}
Недостатками можно назвать:
\begin{list}{}{}
    \item Написана на Java, поэтому требует больше ресурсов, чем аналог на C++
    \item Подключение нового устройства потребует написания дополнительного модуля
\end{list}
