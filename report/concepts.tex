\section{Основные понятия}

Рассмотрим систему домашней автоматизации более формально.
Умный дом -- система домашних устройств, способных выполнять действия и решать определенные повседневные задачи без участия человека.
Поскольку создание децентрализованной системы потребует согласованности производства всех устройств, удобнее будет воспользоваться централизованной системой.
Центральный узел назовём шиной (или контроллером), а бытовые устройства, входящие в эту систему -- периферийными устройствами.
Средства соединения устройств называются каналами или мостами.

Для того, чтобы периферия могла взаимодействовать с шиной, нам потребуется протокол общения.
Большая часть задач не требует изобретения своего протокола, достаточно на стороне шины поддержать адаптер, позволяющий работать с нужным протоколом.
Система вполне может быть гетерогенной и одновременно использовать несколько протоколов.
Например, X10 \cite{X10Spec} для общения с датчиком температуры и REST \cite{RESTSpec} или gRPC \cite{gRPCSpec} поверх HTTPS для соединения с интерфейсом, отображающим показатели системы.

Каналы общения устройств будут находиться на уровне PAN (Personal Area Network) или LAN (Local Area Network).
Эти два уровня подходят для передачи небольших объёмов данных на небольшие расстояния.
При подключении к сетям более высокого уровня умный дом становится частью IoT (Internet of Things).
IoT ({\itshape букв.} интернет вещей) -- общая мировая сеть, объединяющая все устройства, способные подключаться к интернету.
Подключение к такой сети увеличивает число возможностей для управления системой и, как в случае с умными помощниками, позволяет добавить дополнительные функции.
Но, вместе с тем, растёт количество потенциальных угроз, что может быть критично для устройств, управляющих подачей отопления, воды или, тем более, сигнализацией.

Кроме того, умный дом должен уметь выполнять домашнюю работу и задачи, которые ставит ему человек.
К этому разные системы подходят по-разному. Иногда определяют несложный язык для формирования правил и условий, по которым далее будут выполняться действия.
Иногда используют базу -- голосового помощника, как в случае с Google Assistant и Amazon Alexa.
