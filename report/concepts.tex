\section{Основные понятия}

Рассмотрим систему домашней автоматизации более формально.
Умный дом -- система домашних устройств, способных выполнять действия и решать определенные повседневные задачи без участия человека.
Поскольку создание децентрализованной системы потребует согласованности производства всех устройств, удобнее будет воспользоваться централизованной системой.
Центральный узел назовём шиной, а бытовые устройства, входящие в эту систему -- периферийными устройствами.
Средства соединения устройств называются каналами или мостами.

Для того, чтобы периферия могла взаимодействовать с шиной, нам потребуется протокол общения.
Большая часть задач не требует изобретения своего протокола, достаточно на стороне шины поддержать адаптер, позволяющий работать с нужным протоколом.
Система вполне может быть гетерогенной и одновременно использовать несколько протоколов.
Например, X10 для общения с датчиком температуры и REST поверх HTTPS для соединения с интерфейсом, отображающим показатели системы.
