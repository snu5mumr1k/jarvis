\section{Задачи в области}

К настоящему моменту человечество уже сильно продвинулось в реализации умных домов.
Некоторые решения встраиваются в здания при строительстве, тем самым улучшая интеграцию с бытовыми приборами, розетками, отопление и т.д.

Какие же задачи на данный момент являются нерешёнными или решёнными недостаточно оптимально?
\begin{list}{}{}
    \item Обобщить протоколы общения между устройствами. Не существует единого стандарта протокола. Это приводит к тому, что устройства разных производителей труднее совмещать в сложной системе.
    \item Придумать и воплотить стандарты безопасности для умных домов. Поскольку домашняя автоматизация ещё не стала повсеместной, особых требований к подобным системам, пока что, нет. Но, с распространением концепции, они, безусловно, должны появиться.
    \item Довести простоту использования до того уровня, когда людям будут не нужны дополнительные навыки для повседневного использования технологии. Сейчас для использования умного дома, скорее всего, потребуются начальные навыки в программировании.
\end{list}
