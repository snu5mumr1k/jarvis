\section{Введение}

Почти 70 лет прошло с тех пор, как был издан рассказ Рэя Бредбери "There Will Come Soft Rains".
В этом рассказе описан постапокалиптичный мир, в котором продолжает функционировать система домашних устройств, связанных воедино общей программой.
Еда готовится сама к назначенному времени. Уборка в доме, стирка и другие бытовые задачи за человека автоматически решают машины.
Также дом оснащён динамиками для осуществления связи с людьми и проигрывания музыки.
Таков был образ "умного дома" в середине прошлого века. Он почти совпадает с тем, что мы хотели бы видеть сейчас.

Рассмотрим, как развивалась история домашней автоматизации.

Человек на протяжении всей своей истории стремился улучшать условия своей жизни.
Уже в десятом тысячелетии до нашей эры появляются первые зернохранилища.
В третьем тысячелетии до нашей эры в долине реки Инд появилась канализационная система.
В античности было создано по крайней мере два широко известных изобретения, которые можно назвать "автоматизацией быта". Одним из них является архимедов винт, другим -- римский водопровод.
Оба позволили сильно упростить использование воды в доме и автоматизировать процесс её доставки.
За античностью последовали довольно скупые на изобретения годы средневековья.
Следующие крупные шаги в направлении улучшения быта появятся только в начале XIX века и дальше продолжатся до новейшей истории и современности.
В этом периоде можно выделить этап изобретения самих бытовых приборов: \cite{ToolsHistory}
\begin{list}{}{}
\item газовая плита (1826) 
\item холодильник (1834)
\item лампа накаливания (1835)
\item пылесос (1869)
\item электрическая плита (1883)
\item посудомоечная машина (1886)
\end{list}
За ним последовал этап автоматизации и налаживания взаимодействия этих устройств.
Именно он и представляет особый интерес в виду того, что он ещё совсем не завершён и есть задачи, оптимального решения которых пока что нет.

Поскольку автоматизация быта предполагает взаимодействие между различными представителями бытовой техники, логично предположить существование стандартных протоколов общения устройств.
Но это не так. Правильнее сказать, что протоколы есть, но нет стандартизации и общности.
Одним из первых протоколов был X10. Он основан на передаче пары датаграмм, первая из которых кодирует устройство, а вторая -- команду для этого устройства.
В силу своей простоты стал достаточно популярен, но не стал стандартом, в виду того, что не обеспечивает надёжности и безопасности доставки сообщений.
Сейчас протоколов существует достаточно много. Например, Zigbee \cite{ZigBeeDoc} (основан на IEEE 802.15.4 -- одном из стандартов персональных сетей) или Z-Wave \cite{ZWaveDoc} (проприетарный протокол, ставший в последнее время популярным -- 2.5 тысячи различных устройств на рынке).

Также интерес представляет вопрос защиты умного дома от кибератак и построения системы достаточной надёжности, чтобы человек был согласен доверять ей жизнь.

От выбора технологий при построении системы умного дома зависит то, насколько просто будет ей управлять, чинить её и модернизировать.
Так что чем более гибким устройством и чем большей модульностью она будет обладать, тем лучше конечному пользователю.
И не стоит забывать, что, пока эти технологии не так широко распространены, полную систему собрать либо не очень просто, либо дорого.

В качестве решения вышеперечисленных проблем, можно рассмотреть открытый проект openHAB, который основан на фреймворке с открытым исходным кодом Eclipse~SmartHome \cite{EclipseSmartHomeGitHub}.
При его рассмотрении мы не будем сильно углубляться в вопросы безопасности и стоимости.
При продвинутом использовании можно построить недорогую защищённую систему.
Нас будет интересовать устройство умного дома с точки зрения взаимодействия между устройствами.
