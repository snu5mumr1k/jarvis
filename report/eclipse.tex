\section{Фреймворк SmartHome от Eclipse}

Остановимся подробнее на данном фреймворке (под фреймворком здесь будем понимать кодовую базу, реализующую остов нужной функциональности и упрощающую дополнительную разработку).
Его кодовая база отличается от базы Domoticz, кроме языка, ещё объёмом и простотой написанного кода.
Код Domoticz, в силу того, что старше и что вышел до популяризации принципов C++11 содержит устаревшего и потенциально опасного кода (например, прямая работа с указателями и памятью).
Преимуществом проекта Eclipse является то, что рабочее окружение Java разрабатывается отдельно и отдельно улучшается.

Перейдём к самому коду:
При реализации использована технология OSGI \cite{OSGI}, которая позволяет динамически подменять компоненты, не пересобирая приложения полностью.
Таким образом, дополнительно написанные модули могут быть встроены в систему без её перекомпиляции.

Основными концептами кода, которые отражают реальный мир, являются \cite{EclipseDoc}:
\begin{list}{}{}
    \item Вещи (Items) -- это виртуальное представление предметов, их функции и свойства, которые потом могут быть использованы приложением. Например, цвет, электрический контакт, выключатель, строка, число и т.п.
    \item Предметы (Things) -- реальные физические устройства. Могут предоставлять несколько вещей.
    \item Каналы (Channels) -- каналы, которые представляют из себя функции предметов. Например, предмет лампочка будет иметь канал передачи температуры (в формате вещи-цвета) и канал передачи самого цвета как вещи-числа, например
    \item Мосты (Bridges) -- предметы, являющиеся связующими элементами системы. Например, провод, IP-шлюз.
    \item Статусы Предметов (Thing Statuses) -- элемент перечисления статусов, характеризующий состояние предмета в данный момент времени. Более детальная информация в Деталях статусов (Status Details)
    \item Категории (Categories) -- надгруппы объектов. Позволяют вводить функциональные или визуальные обобщающие свойства. Например, для предметов, являющихся дверьми, рисовать одну и ту же иконку в интерфейсе.
\end{list}

В качестве основного паттерна можно выделить паттерн Observer.
Предметы обмениваются событиями (Events) с центральной шиной через мосты.
Центральная шина обеспечивает обработку этих событий по правилам и рассылку уведомлений подписавшися сервисам.
Правила обработки можно задавать с помощью языка с простым синтаксисом, описание которого мы сейчас опустим. Оно есть в документации \cite{EclipseRules}

Сами события представляют из себя набор из четырёх строк:
\begin{list}{}{}
    \item Тип -- тип события. Позволяет приложению подписаться на конкретный тип события.
    \item Топик -- категория событий. Позволяет строить обобщения над событиями и подписываться только на определённые топики.
    \item Сообщение -- собственно, информация. Зависит от вещи-источника.
    \item Источник -- идентификатор вещи-источника.
\end{list}
Поскольку событие имеет такую простую структуру и состоит из строк, его можно передавать по произвольным каналам. Например, через REST API.

Кроме того, создатели фреймворка договариваются с производителями оборудования и генерация событий в нужном формате может происходить прямо в устройстве.
