\section{Возможные улучшения фреймворка}

Как уже было сказано, SmartHome написан на Java, а значит, везде вместе с программой должна быть среда JRE для запуска окружения.
Кроме того, модули можно писать только на Java, что сужает круг возможностей.
Можно было бы написать аналог данного фреймворка, взяв за основу концепт основанной на передаче событий централизованной системы,
но использовать при этом C++, уменьшив при этом объём, занимаемый системой, что позволит портировать её на микроконтроллеры.
Кроме того, можно реализовать (через JIT-компиляцию как в Domoticz) поддержку модулей на Lua, что уменьшит порог входа для людей, далёких от разработки программ, поскольку
Lua -- язык более простой, чем Java и чем C++.
